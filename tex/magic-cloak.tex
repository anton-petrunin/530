\chapter{Magic cloaks}

This lecture is based on a lecture of Sergei Ivanov \cite{ivanov-2012}.

\section{Scattering data}

Given a Riemannian manifold $M$ denote by $\S M$ its unit tangent bundle
\[\S M=\set{\vec v\in \T M}{|v|=1}.\]
Recall that geodesic flow $\phi^t$ preserves the natural volume on $\S M$.

Suppose $M$ has nonempty boundary $\partial M$;
in other words, $M$ is a closed region of a ambient Riemannian  manifold bounded by a smooth hypersurface $\partial M$.
Denote by $\S^+ \partial M$ the set of unit vectors at points on $\partial M$ that point in $M$;
$\S^+ \partial M$ is a bundle over $\partial M$ with fibers formed by closed half-spheres.

Consider a geodesic $\gamma_{\vec u}$ in the direction of a vector $\vec u\in \S^+ \partial M$.
Suppose that $\gamma_{\vec u}$ hits the boundary again; 
denote by $\ell_{\vec u}$ the first \emph{hitting time}, so $\gamma_{\vec u}(\ell_{\vec u})\in \partial M$.
Note that in this case $\vec v=-\gamma'(\ell)\in \S^+ \partial M$.
Note that $\vec u\mapsto \vec v$ describes a partially defined involution on $\S M$ which we will call \emph{scattering map}.

Suppose that $M$ and $\bar M$ be two compact connected Riemannian manifolds with boundary such that a neighborhood of $\partial M$ can be isometrically identified with a neighborhood of $\partial \bar M$,
and moreover, the scattering maps in $M$ and $\bar M$ are identical.
In this case we say that $M$ and $\bar M$ have identical  \emph{scattering data}.
If in addition the hitting time functions coincide then they are said to have the same \emph{lens data}.

Notice that a complement of a small ball in the standard sphere $\SSS^n$ is visually identical to the complement of identical ball in the projective space $\RP^n=\SSS^n/\ZZ_2$.
\begin{figure}[h!]
\centering
\includegraphics[scale=.7]{pics/sphere.png}
\end{figure}

Note that the lens data for the constructed pair of manifolds are not identical.
An example of nonisometric manifolds with identical lens data can be found among surfaces of revolution which look like cylinders with bumps on them that are shifted and otherwise look the same.
\begin{figure}[h!]
\centering
\includegraphics[scale=.7]{pics/revolution.png}
\end{figure}

\begin{thm}{Exercise}
Check that the described examples have identical lens data.

Construct a nonisometric pair of Riemannian metrics on the disc with identical lens data.
\end{thm}

\begin{thm}{Theorem}\label{thm:magic-cloak}
Any connected compact region $M$ of Euclidean space of dimension at least 2 cut by a smooth hypersurface is \emph{scattering rigid};
that is, any manifold with scattering data identical to $M$ is isometric to $M$.
\end{thm}

\begin{thm}{Corollary}
Suppose a Riemannian metric $g$ on $\RR^n$ coincides with Euclidean metric $g_0$ outside of a compact set $K$.
Suppose that that the complement $\gamma\backslash K$ of any $g$-geodesic $\gamma$ coincides with the complement of a line (as sets).
Then $(\RR^n,g)$ is isometric to the Euclidean space.
\end{thm}

Note that one cannot claim that $g=g_0$.
Indeed for any diffeomorphism $\phi\:\RR^n\to\RR^n$ that is identical outside $K$ the metric $g=\phi_*g$ satisfies the assumption in the corollary.
Clearly one can choose $\phi$ so that $\phi_*g_0\ne g_0$.


\section{Lens data}

\begin{thm}{Lemma}\label{lem:no-delay}
Suppose that a manifold $\bar M$ has identical scattering data with a compact region $M$ in Euclidean space of dimension at least 2.
Then $M$ and $\bar M$ have identical lens data. 
\end{thm}

\parit{Proof.}
Note that we can cut $M$ from the Euclidean space $E$ and glue $\bar M$ back in along the isometry in the definition of scattering data.
This way we obtain a complete Riemannian manifold $\bar E$ that is Euclidean outside of a $\bar M$.

Suppose that a smooth hypersurface $\Sigma$ in $E$ surrounds $M$.
Then $\Sigma$ cuts from $E$ and $\bar E$ manifolds with identical scattering data.
Moreover if $\bar M$ and $M$ are not isometric,
then the obtained pair is not isometric as well.

It follows that we can assume that $M$ is a ball.
(This is true for the proof of the lemma and theorem as well.)

In this case any geodesic $\bar\gamma$ in $\bar E$ visits $\bar M$ at most once.
In other words, if $\bar\gamma$ enters $\bar M$, then the complement $\bar\gamma\backslash \bar M$ has two connected components which are parts of a line $\gamma$ in $E$.

Choose a unit-speed geodesic $\bar\gamma$ that visits $\bar M$.
Let include it in a smooth one-parameter family of unit-speed geodescis $\bar\gamma_\tau$ for $\tau\in [0,1]$ so that $\bar\gamma_0$ does not visit $\bar M$ and $\bar \gamma_1=\bar \gamma$.

We can assume that 
the vector field $\bar{\vec i}
=\tfrac{\partial}{\partial\tau}\gamma_\tau(t)$ is orthogonal to $\bar{\vec t}
=\tfrac{\partial}{\partial t}\gamma_\tau(t)$ at every point $\gamma_\tau(t_0)$ for some fixed $t_0$.

Observe that in this case $\langle\bar{\vec i},\bar{\vec t}\rangle=0$ at all points $\bar\gamma_\tau(t)$.
Indeed 
\begin{align*}
\tfrac{\partial}{\partial t}\langle\bar{\vec i},\bar{\vec t}\rangle&=
\bar{\vec t}\langle\bar{\vec i},\bar{\vec t}\rangle=
\\
&=
\langle\nabla_{\bar{\vec t}}\bar{\vec i},\bar{\vec t}\rangle+\cancel{\langle\bar{\vec i},\nabla_{\bar{\vec t}}\bar{\vec t}\rangle}=
\\
&=\langle\nabla_{\bar{\vec i}}\bar{\vec t},\bar{\vec t}\rangle=
\\
&=\tfrac12\cdot\bar{\vec i}\langle\bar{\vec t},\bar{\vec t}\rangle=
\\
&=\tfrac{\partial}{\partial \tau}|\bar\gamma_\tau'(t)|^2=
\\
&=0.
\end{align*}
That is $\langle\bar{\vec i},\bar{\vec t}\rangle$ does not depend on $t$.
Since $\langle\bar{\vec i},\bar{\vec t}\rangle=0$ at all points $\gamma_\tau(t_0)$, the same holds for all points  $\gamma_\tau(t)$.

Consider a family of geodesics $\gamma_\tau$ in $E$ that coincide with $\bar\gamma_\tau$ (as sets) outside of $M$.
The same argument shows that $\langle\vec i,\vec t\rangle=0$ at all points $\gamma_\tau(t)$.
It follows that $\bar{\vec i}=\vec i$ outside of $\bar M$ and $M$ correspondingly.
Therefore $\gamma_\tau(t)=\bar\gamma(t)$ for any $t$ and $\tau$, provided that $\bar\gamma(t)\notin \bar M$.
Whence the $\gamma$ spends exactly the same time in $M$ as 
$\bar \gamma$ spends in $\bar M$ and the lemma follows.\qeds

\parit{Comment.} The vector fields $\vec i$  as in the proof restricted to $\gamma_\tau$ are called \emph{Jacobi fields} along $\gamma_\tau$;
we will see them again.

\begin{thm}{Exercise}\label{ex:exit-time}
Suppose $\bar M$ and $\bar E$ be as in the proof.
Show that there is a universal upper bound on time that a unit-speed geodesic spends in $\bar M$.
\end{thm}

\parit{Hint:} Show that the set of vectors $\vec u\in\S \bar E$ such that $\gamma_{\vec u}([0, T])\subset \bar M$ is open and closed;
here $T=2\cdot\diam M$ and $\gamma_{\vec u}$ is the geodesic defined by $\gamma'_{\vec u}(0)=\vec u$.

\section{Volume equality}

\begin{thm}{Lemma}\label{lem:vol=}
Suppose that $M$ and $\bar M$ have identical scattering data.
Then 
\[\vol M=\vol \bar M.\]

\end{thm}

\parit{Proof.}
Recall that $\bar s\:\S \bar M\to \bar M$ denotes the unit tangent bundle over $\bar M$
and $\bar\phi^t\:\S \bar M\to \S \bar M$ denotes the geodesic flow.

Set $\bar\Omega=\bar s^{-1}(\bar M)$.
Since geodesic flow preserves the volume, we get 
\begin{align*}
\vol\SSS^{n-1}\cdot\vol(\bar M,g)&=\vol\bar\Omega=
\\
&=\vol[\bar\phi^t(\bar\Omega)].
\end{align*}

By \ref{ex:exit-time}, we can choose $t$ so that $s(\vec v)\notin \bar M$ for any $\vec v\in \phi^t(\bar\Omega)$.

Repeat the same construction for $M$. 
By \ref{lem:no-delay}, the set $\phi^t(\Omega)\z=\bar\phi^t(\bar\Omega)$.
In particular, 
\[
\vol\Omega=
\vol[\phi^t(\Omega)]=\vol[\bar\phi^t(\bar\Omega)=\vol\bar\Omega.
\]
whence the result follows.
\qeds

\section{Santalo formula}

Santalo formula gives an expression for a volume of a Riemannian manifold with boundary in terms of hitting times of its geodesics.
It provides a more direct proof of \ref{lem:vol=}.

Suppose $M$ is a Reimannian manifold with nonempty boundary $\partial M$.
Recall that 
\begin{itemize}
\item $\S M$ denotes the unit tangent bundle of a Reimannian manifold $M$.
\item $\S^+ \partial M$ denotes by the set of unit vectors at points on $\partial M$ that point in $M$.
\item $\ell\:\S^+ \partial M\to [0,\infty]$ denoted the hitting time.
\end{itemize}

\begin{thm}{Theorem}
Let $M$ be an $n$-dimensional Riemannian manifold with nonempty boundary.
Suppose that any unit-speed geodesic in $M$ hits its boundary in finite time.
Then for any smooth function $f\:\S M$ the following identity holds:
\[\int_{\vec w\in \S M}f(\vec w)
=
\int_{\vec u\in \S^+ \partial M} 
\int_0^{\ell(\vec u)} f[\gamma'_{\vec u}(t)]\cdot\langle\vec u,\vec n\rangle\cdot dt,\]
where $\vec n$ denotes the unit vector field normal to $\partial M$ that points inside~$M$

In particular, by taking $f\equiv 1$, we get
\[\vol\SSS^{n-1}\cdot\vol M=\int_{\vec u\in \S^+ \partial M} \ell(\vec u)\cdot\langle\vec u,\vec n\rangle.\]

\end{thm}

\begin{thm}{Exercise}
Construct two Reimannian metrics $g_0$ and $g_1$ on the disc $\DD$ that coincide near the boundary and such that 
\[\area (\DD,g_0)>\area (\DD,g_1),\]
but 
\[\ell_0(\xi)<\ell_1(\xi),\]
where $\ell_i(\xi)$ denotes hitting time of $g_i$-geodesic in the direction $\xi$;
that is, 
$\ell_i(\xi)$ is the length of $g_i$-geodesic that starts at a point $p\in \partial \DD$ in the direction $\xi$.

Why does this example  not contradict Santalo's formula?
\end{thm}

\section{More exercises}

Two Riemannian metrics $g_0$ and $g_1$ on $M$ are called \emph{conformally equivalent} if there is a function $\lambda$ such that $g_1=\lambda^2\cdot g_0$.
In this case the function $\lambda$ is called \emph{conformal factor}.
Note that for any $g_0$-unit-speed curve $\gamma\:[a,b]\to M$ 
we have
\[\length_{g_1}\gamma=\int_a^b\lambda\circ\gamma(t)\cdot dt\]

\begin{thm}{Exercise}
Let $g_0$ be the canonical metric on the projective space $\RP^n$; 
that is, $(\RP^n,g_0)$ is isometric to the quotient space of the unit sphere $\SSS^n$ by central symmetry.
Suppose that $g_1$ is conformally equivalent to $g_0$.
Denote by $\ell_0$ and $\ell_1$ the length of shortest noncontractible closed curves in $(\RP^n,g_0)$ and $(\RP^n,g_1)$ respectively (so $\ell_0=\pi$).
Show that 
\[\frac{\vol(\RP^n,g_1)}{\ell_1^n}\ge \frac{\vol(\RP^n,g_0)}{\ell_0^n}.\]
\end{thm}

\parit{Hint:} Use that geodesic flow preserves volume of the unit tangent bundle to rewrite the integral of conformal factor over $(\RP^n,g_0)$ and interpret the result.

\bigskip

Let $(M,g)$ be a Riemannian manifold.
The Sasaki metric is a natural choice of Riemannian metric $\hat g$ on the total space of the tangent bundle $\tau\:\T M\to M$ defined the following way:

Identify the tangent space 
$\T_u[\T M]$ for any $u\z\in \T_p M$ with the direct sum of vertical and horizontal subspaces $\T_p M\z\oplus \T_p M$.
The projection of this splitting is defined by the differential $d\tau\:\T\T M\to \T M$
and we assume that the velocity of a curve in $\T M$ formed by a parallel field along a curve in $M$ is horizontal.
Then $\T_u[\T M]$ is equipped with the metric $\hat g$ defined by
\[\hat g(X,Y)=g(X^V,Y^V)+g(X^H,Y^H),\]
where $X^V$ and $X^H\in\T_pM$ denote the vertical and horizontal components of $X\in\T_u[\T M]$.



\begin{thm}{Exercise}
Let $g$ be the canonical Riemannian metric on the sphere $\mathbb{S}^2$.
Consider the tangent bundle $\T \mathbb{S}^2$ 
equipped with the induced Sasaki metric $\hat g$.
Let $S_R$ be the hypersurface in $\T \mathbb{S}^2$ of vectors with norm $R$;
we assume that $S_R$ is equipped with induced Riemannian metric.

Show that $\vol S_R\to \infty$ as $R\to\infty$,
but $\diam S_R$ stays bounded for all $R$.
\end{thm}



\section{Remarks}

The fact that not all manifolds are scattering rigid was pointed out by Christopher Croke \cite{croke-1991}.
More examples constructed by Christopher Croke and Bruce Kleiner \cite{croke-kleiner-1994}.

Theorem \ref{thm:magic-cloak} proved by Mikhael Gromov \cite{gromov-1983}.
It has a number of variations and generalizations.
In particular an analog of this theorem holds in the following cases:
\begin{itemize}
\item For 2-dimensional Riemannian manifolds with unique geodesic between any two points \cite{pestov-uhlmann}.
\item In hyperbolic space \cite{besson-courtois-gallot-1995, } and for  regions in a round  hemisphere \cite{michel-1981}.
\item If $M$ is a Riemannian manifold with unique geodesic between any two points, then the theorem holds for the product $\RR\times M$ \cite{croke-kleiner-1998}.
\item For any Riemannian manifold, provided that the metric tensor is modified to in a sufficiently small region \cite{burago-ivanov-2010}.
\end{itemize}



