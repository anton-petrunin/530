\documentclass[a4paper,10pt]{article}
\usepackage{epsfig,lpic,wrapfig}
%\usepackage[utf8]{inputenc}
%\usepackage[russian]{babel}
\usepackage[active]{srcltx}
\usepackage{amsmath,amssymb} 
\usepackage{textpos}
\usepackage{amssymb, amsfonts, amsmath, amsthm}
\usepackage{enumerate}
\usepackage{bm}
\usepackage{color}
\usepackage{pifont}
\usepackage{enumerate}

\def\CC{\mathbb{C}}%  
\def\PP{\mathbb{P}}%  
\def\NN{\mathbb{N}}%  
\def\RR{\mathbb{R}}%  
\def\ZZ{\mathbb{Z}}% 
\def\QQ{\mathbb{Q}}%  
\def\T{\mathrm{T}}%
\def\eps{\varepsilon}% 
\def\vol{\mathrm{vol}}% 
\newcommand*{\tc}[1]{\Phi({#1})}
\def\length{\mathop{\rm length}\nolimits}%  
\def\"#1{{\accent"7F #1\penalty10000\hskip 0pt plus 0pt}} % ! % 
\def\:{\colon}
\def\kur{k}
\def\tor{\kappa}


\begin{document}
\pagestyle{empty}
\begin{textblock*}{150mm}(.0\textwidth,-4cm)
\hfill
Math 530, HWA-09
\end{textblock*}

\vskip-15mm

-. Let $\phi$ be a flow on a Riemannian manifold $(M,g)$ generated by a vector field $X$. 
Assume that $\phi^t$ is an isometry for each $t$.
Show that the equality
\[\langle\nabla_YX,Z\rangle+\langle Y,\nabla_ZX\rangle=0\]
holds for any two vector fields $Y$ and $Z$ on $M$.

\medskip
\noindent\textit{Comment.} A field $X$ as above is called a \emph{Killing field}.


\ 

-. Let $\hat M$ be a submanifold of a Riemannian manifold $M$.
Note that $\hat M$ comes with a metric tensor, induced from $M$; in particular, $\hat M$ is a Riemannian manifold.
Denote by $\nabla$ and $\hat\nabla$ the Levi-Civita connections on $M$ and $\hat M$ respectively.

Show that for any vector fields $\hat X$ and $\hat Y$ on $\hat M$ we have that
\[S(\hat X,\hat Y)=\hat\nabla_{\hat X}{\hat Y}-\nabla_{\hat X}{\hat Y}\]
is perpendicular to $\hat M$;
that is, $\langle S(\hat X,\hat Y),\hat W\rangle\equiv0$ for any tangent vector field $\hat W$ on~$\hat M$.

Show that for any $p\in \hat M$, the value $[S(\hat X,\hat Y)](p)$ depends only on $\hat X(p)$ and~$\hat Y(p)$.

\medskip
\noindent\textit{Comment.} $S$ can be called \emph{second fundamental form} of $\hat M$ in $M$;
it is a bilinear form on the tangent bundle of $\hat M$ with values in the normal bundle of $\hat M$ in $M$.
Usually second fundamental form is defined for hypersurfaces; it has values in $\RR$ --- for hypersurfaces it can be identified with the normal bundle (locally).


\ 

-. A function $f$ on a Riemannian manifold is called convex if for any geodesic $\gamma$, the composition $f\circ\gamma$ is a convex real-to-real function.
Assume a complete Riemannian manifold $(M,g)$ admits a nonconstant convex function.
Show that $\vol(M,g)=\infty$.

\medskip
\noindent\textit{Comment.}
A Riemannian manifolds is complete if the corresponding metric space is complete.
This condition is equivalent to the fact that there is a both-side infinite geodesic in any direction.

\ 

-. Let $M$ and $N$ be complete $m$-dimensional simply connected Riemannian manifolds, and $f\:M\to N$ a smooth map such that 
$$|df(V)|\ge |V|$$
for any tangent vector $V$ of $M$.
Show that $f$ is a diffeomorphism.

\ 

-. Let $(M,g)$ be a closed $n$-dimensional Riemannian manifold and $f\:M\to\RR$ is a 1-Lipschitz function.
Consider equally distributed random vector $V$ with norm at most $1$; denote by $\tau(V)$ its base point.
Show that the expected value of 
\[|f(\exp(V)-f(\tau(V))|\le \eps_n,\] 
where $\eps_n$ depends only on the dimension $n$ and $\eps_n\to 0$ as $n\to\infty$.


\medskip
\noindent\textit{Comment.}
This problem  is an example in the Riemannian world of the so-called 
}\emph{concentration of measure phenomenon}.

Prove first the following: Suppose $V$ is an equally distributed random vector in $\RR^n$ such that that $|V|\le 1$.
Show that the expected absolute value of its first coordinate converges to 0 as $n\to \infty$.


\end{document}
