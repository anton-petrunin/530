




\begin{thm}{Exercise}\label{ex:Rop-Li}
Show that curvature operator of an $n$-dimensional Lie group with biinvariant metric has rank at most $n$ and nonnegative.

Conclude that if a compact Lie group with biinvariant metric has unit sectional curvature in all directions, then it has dimension at most~3.
\end{thm}

\parit{Hint:} Note that  $A\:\vec x\wedge \vec y\mapsto [\vec x,\vec y]$ is defined and it can be extended to a liner map $A\:\Lambda^2\mathfrak{g}\to\mathfrak{g}$ (here $\mathfrak{g}=\T_eG$ --- the tangent space at the unit element $e\in G$).
Furhter, show and use that (1) the rank of $A$ is at most $n$ and $\Rop =\alpha(A^*\cdot A)$, where $\alpha$ denotes complete antisymmetrization.


















Consider two symmetric quadratic forms $P,Q\in\S^2\T$.
The following expression 
\begin{align*}
P\KN Q(\vec x,\vec y,\vec v,\vec w)
&=
P(\vec x,\vec v)\cdot Q(\vec y,\vec w)-P(\vec x,\vec w)\cdot Q(\vec y,\vec v)\\
&+Q(\vec x,\vec v)\cdot P(\vec y,\vec w)-Q(\vec x,\vec w)\cdot Q(\vec y,\vec v).
\end{align*}
is called Kulkarni--Nomizu product.

\begin{thm}{Exercise}
Show that $P\KN Q\in \A^4\T$ for any pair  $P,Q\in\S^2\T$.

Show that a curvature tenor $R$ has constant sectional curvature $\kappa$ if and only if 
\[\hat \Rm=-\tfrac\kappa2\cdot g\KN g.\]
\end{thm}
