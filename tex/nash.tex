\chapter{Fundamental theorem}

This lecture is based on a tiny piece from the book by Mikhael Gromov~\cite{gromov-1986}.

\section{Formulation}

The name \emph{Fundamental theorem of Riemannian geometry} can be used for two results: the theorem on existence and uniqueness of Levi-Civita connection on Riemannian manifold and the following theorem proved by John Nash \cite{nash-1956}:

\begin{thm}{Fundamental theorem}\label{thm:nash}
Any $n$-dimensional Riemannian manifold $(M,g)$ admits a smooth length-preserving embedding into a Euclidean space of sufficiently large dimension $q$.
\end{thm}

We will prove this result modulo the so-called  Nash--Moser implicit function theorem.
We will assume that $M$ is compact, but it is not all a principle assumption.

The dimension $q$ can be found explicitly in terms of $n$.
For example, any $q=100\cdot n^2$ will do, but we will only show that there is $q$ that depends on $M$.


\section{Induced metric}

Recall that a field $g$ of bilinear forms on the $M$ is called \emph{metric tensor}.
A metric tenor $g$ is called \emph{Riemannian} if it is positive definite;
that is, $g(\vec v,\vec v)>0$ for any $\vec v\ne 0$.

Let $\bm{f}\:M\to\RR^q$ be a smooth map defined on a manifold $M$; here we consider $\RR^q$ with standard Euclidean metric.
We say that a metric tenor $g$ is induced by $\bm{f}$ if
\[g(\vec v,\vec w)=\langle \vec v\bm{f},\vec w\bm{f}\rangle,
\quad\text{or, equivalently}\quad
g(\vec v,\vec w)=\langle (d\bm{f})\vec v,(d\bm{f})\vec w\rangle\]

Note that $\bm{f}\:(M,g)\to\RR^q$ is length-preserving if and only if 
\[g(\vec v,\vec v)=|\vec v\bm{f}|^2\]
for any tangent vector $\vec v$.

Recall that any bilinear form $g$ completely determined by the corresponding quadratic form; 
that is, if we know $g(\vec v,\vec v)$ for any vector $\vec v$, then we know $g(\vec v,\vec w)$ for any pair of vectors $\vec v,\vec w$.
The latter is proved by the following identity:
\[g(\vec v,\vec w)=\tfrac12\cdot\left[ g(\vec v+\vec w,\vec v+\vec w)-g(\vec v,\vec v)-g(\vec w,\vec w)\right].\]
Therefore $\bm{f}\:(M,g)\to\RR^q$ is length-preserving if and only if
\[g(\vec v,\vec w)=\langle \vec v\bm{f},\vec w\bm{f}\rangle.\]

Assume that $f_1,\dots,f_q$ are coordinate functions of $\bm{f}$.
Then the latter identity can be written as
\[g=(df_1)^2+\dots+(df_q)^2,\]
where $(df_i)^2$ is a shortcut for the metric tenor $b_i$ defined by
\[b_i(\vec v,\vec w)\df df_i(\vec v)\cdot d f_i(\vec w)=(\vec vf_i)\cdot (\vec wf_i).\]


\[g=(df_1)^2+\dots+(df_q)^2.\]

Let us show that the fundamental theorem can be reduced to the following statement (as always, in the compact case).

\begin{thm}{Reformulation}\label{thm:nash-reformulation}
For any Riemannian metric $g$ on a compact smooth manifold $M$ there are smooth functions $f_1,\dots,f_q\:M\to\RR$ such that 
\[g=(df_1)^2+\dots+(df_q)^2.\eqlbl{eq:g=(df)^2}\]
\end{thm}



\parit{Proof of equivalence in the compact case.}
If $\bm{f}\:(M,g)\to \RR^q$ is a smooth length-preserving map, then, as it was shown above, $g$ is induced by $\bm{f}$ and \ref{eq:g=(df)^2} holds for the coordinate functions $f_1,\dots, f_q$ of $\bm{f}$.

Now, assume the reformulation (\ref{thm:nash-reformulation}) is proved.
Consider a smooth embedding $\bm{h}\:M\to \RR^{2\cdot n+1}$ provided by the Whitney embedding theorem.
Denote by $g_0$ the Riemannian metric on $M$ induced by $\bm{h}$;
that is,  
\[g_0=(dh_1)^2+\dots+(dh_{2\cdot n+1})^2,\]
where $h_1,\dots,h_{2\cdot n+1}$ are coordinate functions of $\bm{h}$.
Passing to an scaled embedding $\eps\cdot \bm{h}$ for some small $\eps>0$, we can assume that $g>g_0$; that is, $\bar g=g-g_0$ is a Riemannian metric on $M$.
(Here we used compactness of $M$, but \emph{not} in an essential way.)

Applying the reformulation (\ref{thm:nash-reformulation}) to $(M,\bar g)$ we get a smooth length-preserving immersion $\bm{f}\:(M,\bar g)\to\RR^q$.
It remains to observe that the smooth embedding $M\to \RR^{2\cdot n+1}\oplus\RR^q$ defined by
$x\mapsto (\bm{h}(x),\bm{f}(x))$ has induced metric tensor $g=g_0+\bar g$;
therefore it is length-preserving.
\qeds

\section{Nash's twist}

The following exercise is a weaker form of \ref{thm:nash-reformulation}; it will play a key role in this section.

\begin{thm}{Exercise}\label{ex:nash}
Show that for any Riemannian metric $g$ on a smooth compact manifold $M$ there are smooth functions 
\[\phi_1,\dots,\phi_q,f_1,\dots,f_q\:M\to\RR\]
such that 
\[g=(\phi_1)^2 \cdot (df_1)^2+\dots+(\phi_q)^2\cdot (df_q)^2.\]

\end{thm}

Let $\phi$ and $f$ be smooth functions on a smooth manifold $M$.
Given $\eps>0$, denote by $\SSS^1_\eps$ the circle of radius $\eps$ in $\RR^2$;
consider an length-preserving map $\ell_\eps\:\RR\to\SSS^1_\eps$, say 
\[\ell_\eps(x)=(\eps\cdot\cos\tfrac x\eps,\eps\cdot\sin\tfrac x\eps).\]
Then the map $F\:M\to \RR^2$ defined by
\[F(x)=\phi(x)\cdot (\ell_\eps\circ f(x))\]
is called \emph{Nash's twist} for the triple $(\eps,\phi,f)$.

Suppose that $\vec v$ is a tangent vector on $M$, then
\begin{align*}
\vec vF&=\vec v(\phi\cdot \ell_\eps\circ f)=
\\
&=(\vec v\phi)\cdot(\ell_\eps\circ f) + \phi\cdot (\ell_\eps'\circ f)\cdot(\vec vf)=
\\
&=d\phi(\vec v)\cdot (\ell_\eps\circ f)+\phi\cdot df(\vec v)\cdot (\ell_\eps'\circ f).
\end{align*}

Observe that $|\ell_\eps|=\eps$, $|\ell'_\eps|=1$, and $\ell_\eps\perp \ell'_\eps$.
\[\langle \vec vF,\vec wF\rangle =\eps^2\cdot d\phi(\vec v)\cdot d\phi(\vec w)+\phi^2\cdot df(\vec v)\cdot df(\vec w).\]
Whence we get the following:

\begin{thm}{Claim}\label{clm:twist}
The metric tensor $\phi^2\cdot (df)^2+\eps^2\cdot(d\phi)^2$
is induced by a Nash's twist for $(\eps,\phi,f)$.

\end{thm}


\begin{thm}{Approximation theorem}\label{thm:approx}
Let $(M,g)$ be a compact Riemannian manifold.
Then there are smooth functions $\phi_1,\dots \phi_q$ on $M$ such that 
for any $\eps>0$ the metric tensor 
\[h=(d\phi_1)^2+\dots+(d\phi_q)^2\]
the following condition holds:

For any $\eps>0$, the metric tensor $g+\eps^2\cdot h$ is induced by a smooth map $\bm{F}_\eps\:M\to \RR^{2\cdot q}$ to a Euclidean space.
\end{thm}

\parit{Proof.}
Let $\phi_1,f_1,\dots ,\phi_q,f_q$ be the functions on $M$ provided by \ref{ex:nash}.

Choose $\eps>0$. 
Consider the Nash's twist $F_i$ for each triple $(\eps,\phi_i,f_i)$.
Denote by $\bm{F}_\eps$ the map $M\to\RR^{2\cdot q}$ with pairs of coordinate functions as in $F_1,\dots, F_q$.

By \ref{clm:twist}, the metric tensor $g+\eps^2\cdot h$ is induced by $\bm{F}_\eps$.
\qeds

\section{Pseudoeuclidean case}

Denote by $\RR^{r,s}$ the pseudoeuclidean space with signature $(r,s)$;
that is the space $\RR^{r+s}$ with scalar product defined by
\[\langle\bm{x},\bm{y}\rangle=x_1\cdot y_1+\dots+x_s\cdot y_s-x_{s+1}\cdot y_{s+1}-\dots-x_{s+r}\cdot y_{s+r},\]
where $x_i$ and $y_i$ denote the coordinates of vectors $\bm{x}$ and $\bm{y}$ in $\RR^{r+s}$.

The induced metric tensor for a map to a pseudoeuclidean space can be defined the same way.

\begin{thm}{Theorem}
Any metric tensor $g$ on a compact smooth manifold $M$ is induced by a smooth map $\bm{f}\:M\to \RR^{r,s}$ for some positive integers $r$ and $s$;
in other words, 
\[g=(df_1)^2+\dots+(df_r)^2-(df_{r+1})^2+\dots+(df_{r+s})^2\]
for some smooth functions $f_1,\dots,f_{r+s}$ on $M$.
\end{thm}

\parit{Proof.}
Note that any metric tensor on $M$ can be expressed as a difference of two Riemannian tenors.
Therefore we can assume that $g$ is Riemannian.

Suppose that $\bm{F}_\eps\:M\to \RR^{2\cdot q}$ and $\phi_1,\dots,\phi_{q}$ are provided by the approximation theorem (\ref{thm:approx}).
Consider the map $\bm{\phi}\:M\to \RR^q$  with coordinate functions $\phi_1,\dots,\phi_q$.

Consider the map $\bm{f}\:M\to \RR^{2\cdot q}\oplus \RR^{q}=\RR^{2\cdot q,q}$
defined by
$\bm{f}\:x\z\mapsto (\bm{F}_{\eps}(x),\eps\cdot\bm{\phi}(x))$.
Its induced metric tensor is $g=g+\eps\cdot h\z-\eps\cdot h$.
\qeds

\begin{thm}{Exercise}\label{ex:gromov-p222}
Let $\bm{f}\:(M,g)\to \SSS^{q-1}$ be a smooth length-preserving embedding.
Construct a smooth length-preserving embedding of any conformally equivalent manifold into $\RR^{q,1}$.

That is, given a smooth positive function $\phi$ on $M$, construct a smooth map $\bm{F}\:M\to \RR^{q,1}$
with induced metric tensor $\phi^2\cdot g$.
\end{thm}


\section{Free maps}

Let $\bm{f}\:M\to \RR^q$ be a smooth map defined on a smooth $n$-dimensional manifold $M$.

Recall that $\bm{f}$ is called \emph{regular} if $d\bm{f}$ has rank $n$ at each point.
In other words, for any local coordinates $(x_1,\dots,x_n)$ on $M$ all first order partial derivatives
\[\tfrac{\partial}{\partial x_1}\bm{f},\dots,\tfrac{\partial}{\partial x_n}\bm{f}\]
are linearly independent at each point $p\in M$.

A map $\bm{f}\:M\to \RR^q$ is called \emph{free} if an analogous property holds for first \emph{and second} partial derivatives;
that is, if all $\tfrac{n\cdot(n+3)}{2}$ vectors
\[\tfrac{\partial}{\partial x_1}\bm{f},
\dots,
\tfrac{\partial}{\partial x_n}\bm{f},
\tfrac{\partial^2}{\partial x_1^2}\bm{f},
\tfrac{\partial^2}{\partial x_1\partial x_2}\bm{f},
\dots,
\tfrac{\partial^2}{\partial x_n^2}\bm{f}\]
are linearly independent at each point $p\in M$.
Observe that any free map is regular.

\begin{thm}{Exercise}\label{ex:free-def}
Show that the definition of free map does not depend on the choice of local coordinates. 
\end{thm}

\begin{thm}{Exercise}\label{ex:free-twists}
Consider the $(x,y)$-plane $\RR^2$.
Let $F_x$, $F_y$, and $F_{x+y}$ are Nash's twists $\RR^2\to \RR^2$ for the triples $(1,1,x)$, $(1,1,y)$, and $(1,1,x\z+y)$.
Show that the map $\RR^2\to \RR^6=\RR^2\oplus\RR^2\oplus\RR^2$ defined by $p\z\mapsto (F_x(p), F_y(p),F_{x+y}(p))$ is free.

Generalize the statement to maps $\RR^n\to \RR^{n\cdot(n+1)}$. 
\end{thm}

\begin{thm}{Exercise}\label{ex:free-composition}
Let $\bm{f}\:M\hookrightarrow \RR^q$ is a regular smooth embedding and $\bm{F}\:\RR^q\z\hookrightarrow \RR^Q$ is a free smooth embedding.
Show that the composition $\bm{F}\circ\bm{f}\:M\hookrightarrow \RR^Q$ is free.

Use \ref{ex:free-twists} to conclude that any smooth manifold admits a free embedding into a Euclidean space.
\end{thm}




If $\bm{f}\:M\to \RR^q$ is a smooth embedding, then the smooth manifold $M$ with its image $\bm{f}(M)$.
If $\bm{f}$ is free, than we say that $M$ is a \emph{free submanifold}.

The space spanned by the first and second partial derivatives of $\bm{f}$ at $p$ will be denoted by $\T^2_p=\T^2_pM$.
The ortogonal complement of the tangent space $\T_p$ in $\T^2_p$ will be called \emph{binormal space} and denoted by $\BN_p=\BN_pM$; in other words,
\[\BN_p=\T^2_p\cap\N_p,\]
where $\N_p$ denotes the normal space to $M$ at $p$.

Recall that second fundamental form $S$ is a field of symmetric quadratic forms on $\T M$  with values in $\N M$ that defined by 
\[S(\vec v,\vec w)=\nabla_{\vec v}\vec w-\bar\nabla_{\vec v}\vec w,\]
where $\nabla$ and $\bar \nabla$ denote the Levi-Civita connection on $M$ and the ambient manifold;
in this particular case, $\bar \nabla$ is defined by the parallel translations on the Euclidean space.
Observe that the values of $S$ lie in binormal bundle $\BN M$.

\begin{thm}{Exercise}\label{ex:binormal}
Suppose that $M$ is a free submanifold of $\RR^q$.
Show that for any metric tensor $h$ on $M$ there is a unique binormal field $\vec n$ such  that 
\[h(\vec v,\vec w)=\langle S(\vec v,\vec w),\vec n\rangle\]
for any vector fields $\vec v,\vec w$ on $M$.

Given $h$, consider the one-parameter family of maps $\bm{f}_t\:M\to \RR^q$ defined by 
\[\bm{f}_t(p)= p+t\cdot \vec n(p);\]
let $g(t)$ be the metric tensor induced by $\bm{f}_t$.
Show that $g'(0)=h$.
\end{thm}

The exercise says that a free embedding can be perturbed so that the induced metric tensor moves in a given direction $h$.
Note that freeness of embedding is an \emph{open condition};
namely, if $M$ is a free compact submanifold then any $C^2$-close embedding of $M$ is free as well.
One may think that these two properties \emph{easily} imply the following statement,
but that is not at all easy;
it is a consequence of a deep result --- the so-called Nash--Moser theorem \cite{moser-1961}.
A simplified proof was obtained by Matthias G\"{u}nther \cite{gunter}.

\begin{thm}{Perturbation theorem}\label{thm:perturbation}
Let $\bm{f}\:M\hookrightarrow\RR^q$ be a free embedding, $g$ is the Riemannian metric induced by $\bm{f}$ and $h$ is another metric tensor on $M$.
Then for any $t$ sufficiently close to $0$, there is a free embedding of $\bm{f}_t\:M\hookrightarrow\RR^q$ with  induced metric tensor $g+t\cdot h$.  
\end{thm}

\parit{Proof of \ref{thm:nash-reformulation} modulo the perturbation theorem.}
Choose a free embedding $\bm{f}\: M\hookrightarrow\RR^s$;
it exists by \ref{ex:free-composition}.
Denote by $g_0$ its induced metric.

Scaling down $\bm{f}$ if necessary, we can assume that $g>g_0$;
that is the metric tensor $\bar g= g-g_0$ is Riemannian.

Applying the approximation theorem (\ref{thm:approx}) we get a one parameter family of maps $\bm{F_\eps}\:M\to \RR^q$ with induced metrics $\bar g+\eps^2\cdot h$ for a fixed metric tensor $h$.

By the perturbation theorem (\ref{thm:perturbation}) there is a one parameter family of embedding $\bm{f}_t\:M\to \RR^s$ with induced metric $g_0+t\cdot h$.

Choose sufficiently small $\eps>0$ so that $\bm{f}_t$ is defined for $t=-\eps^2$.
Consider the map $M\to \RR^s\times\RR^q$ defined by 
$x\mapsto (\bm{f}_t(x),\bm{F}_\eps(x))$.
Observe that the induced metric of this map is
\[g_0+t\cdot h+\bar g+\eps^2\cdot h=g.\]
\qedsf

\section{Remarks}

Let us state another closely related result that shows a huge difference between $C^1$ and $C^2$ isometric embeddings.
For example, it implies that the unit sphere admits $C^1$ length-preserving embedding into an arbitrarily small ball in Euclidean 3-space.
There is no such $C^2$-embedding since Gauss curvature of the unit sphere is 1, but at an extremal point it must be at least $\tfrac1{r^2}$, where $r$ is the radius of the ball.

\begin{thm}{Nash--Kuiper theorem}
Let $(M,g)$ be a $n$-dimensional Riemannian manifold and $f\:(M,g)\to \RR^q$ be a short smooth regular map.
Suppose that $q\ge n+1$.
Then for any $\eps>0$ there is an $C^1$-smooth length-preserving maps $f_\eps\:(M,g)\to \RR^q$ that is $\eps$ close to $f$;
that is, $|f_\eps(x)-f(x)|<\eps$ for any $x\in M$.

Moreover if $f$ is an embedding then we can assume that so is $f_\eps$.

\end{thm}


It was originally proved by John Nash \cite{nash-1954} with the condition $q\z\ge n+2$ and improved to $q\ge n+1$ by Nicolaas Kuiper \cite{kuiper-1955}.
The original proof uses Nash's twist in a different way. Both papers are reader-friendly, it is better to start with the paper of Nash.
One may also start with lectures by Allan Yashinski and the author \cite{petrunin-yashinski} where related results were obtained using an alternative approach.

The discussed result formed a part of foundations of the so-called \emph{homotopy principle}, or \emph{h-principle}; an excellent introduction is given in the book by Yakov Eliashberg and Nikolai Mishachev \cite{eliashberg-mishachev}.

Many related questions are open.
For example, it is unknown if \emph{a neighborhood of any point in 2-dimensional Riemannian manifold admits a smooth length-preserving embedding into $\RR^3$}. 






